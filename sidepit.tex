\documentclass[12pt]{article}
\usepackage[utf8]{inputenc}
% \usepackage[english]{babel}
%\usepackage{helvet}
\usepackage{graphicx}
\usepackage{hyperref}
\usepackage{natbib}
\usepackage[inline, shortlabels]{enumitem}
\RequirePackage[a4paper,top=3.5cm,left=3cm,right=3cm,bottom=2.4cm]{geometry}
\title{Fair Price Discovery with Decentralized Exchange}
\author{Yehuda Jay Berg \\jaybny@gmail.com}
\date{May 2021}

\begin{document}
\parindent 0cm
\parskip   6pt
\maketitle

% \documentclass[a4paper]{article}

% \setlength\parindent{0pt}

% \begin{document}

% \title{Sidepit}
% \maketitle
\begin{abstract}
Ever  since  Satoshi  solved  peer  to  peer  digital  cash  with  Bitcoin,  people have  been  trying  to  apply  similar  techniques  to  solve  other  hard  problems. One such problem, peer to peer exchange, is one of the most difficult of these problems. An exchange is a financial market, where trading of securities occur. The purpose of an exchange is two fold;  1) for price discovery, 2) for counterparty settlement. Centralized Exchanges, have been well researched and developed in traditional finance for well over a century.  They evolved from trading under a tree, to the Chicago trading pits, to electronic exchanges with continuous limit order books. Modern exchanges provide 24/7 trading, and offer co-location for the most prolific traders, high frequency trading bots. Other decentralized exchanges aim to bring traditional centralized exchanges into a peer to peer blockchain protocol. Due to early bitcoin exchange hacks most research has been focused on the settlement utility of exchange. We focus on the price discovery utility, an emergent property of real-time trading. We present fair price discovery with decentralized exchange by solving the impossible task of consensus on total ordering of transactions to mitigate front-running.
\end{abstract}

\section{Introduction}
Ever since Satoshi solved peer to peer digital cash with the Bitcoin protocol, people have been trying to apply similar techniques to other hard problems. Peer to peer decentralized exchange (DEX), is one of the most difficult of these problems. 

The purpose of an exchange in financial markets, is to first match buyers and sellers of financial assets at a \emph{market} price, and then to provide settlement to mitigate any counterparty risk. Tradition electronic exchanges are designed with the goal of reaching a  \emph{price discovery} equilibrium \cite{francioni_schwartz_2017}. Centralized exchanges (CEXs), employ continuous limit order books as a price discovery mechanism. 

Due to well known centralized exchange hacks, coupled with the technical difficulty of decentralized limit order books, most DEXs are solely focused on the non-custodial settlement, rather than \emph{price discovery}. 

\paragraph*{Price discovery} is a social benefit and a key goal in the design of a market structure. In fact, the goal of the architecture of an exchange mechanism, is to attract as much liquidity needed to for price discovery.  \cite{francioni_schwartz_2017}

Price discovery is described in microstructure research as a search for an equilibrium price based on new external information. This new information is reflected in the traders orders, and is ultimately converted into a market price. \citep{RePEc:nbr:nberwo:6257}

Some even define an exchange as ``any trading facility that has as its primary function the delivery of good price discovery'' \cite{francioni_schwartz_2017}. 

However, the price discovery function of an  exchange typically receives insufficient attention in market structure discussions. This is largely attributable to the non-observability of equilibrium prices and, therefore, to the difficulty of quantifying the deviations of transaction prices from their equilibrium values \cite{francioni_schwartz_2017}


\begin{quote}
    price discovery is dynamic in nature, and an efficient price discovery process is characterized by the fast adjustment of market prices from the old equilibrium to the new equilibrium with the arrival of new information \cite{RePEc:udb:wpaper:uwec-2005-01-r}    
\end{quote}

\paragraph*{Limit order books} and price discovery are tightly related. \citep{RePEc:nbr:nberwo:6257} \cite{RePEc:eee:jfinec:v:17:y:1986:i:1:p:5-26}

To achieve price discovery exchanges offer two order types. Limit orders, and Market orders. All orders are sent to a centralized matching engine in the exchange servers. 

Market orders have a quantity but no price. 

\begin{enumerate*}
    \item "Buy 1 @ market" - an order to buy 1 unit of the asset at the market  
\end{enumerate*}

Limit orders have a quantity and a price. 

\begin{enumerate*}
    \item "Buy 1 @ 100" - an order to buy 1 unit of the asset at the market  
\end{enumerate*}


\paragraph*{Continuous limit order books (CLOB)} are the market micro structure that leads to price discovery \footnote{Other types of markets such as call auctions, and dealer markets, do not provide the robustness of limit orders for price discovery. \cite{RePEc:hal:journl:hal-00459785}}. There are two order types. Limit orders, where you provide your own price, with the risk of waiting to be matched. And market-orders, where you get filled immediately in return for a possible worse price.

The interaction between market orders matching limit orders is continuous. 

The state of a CLOB in a centralized exchange is shown to produce a price discovery equilibrium. 

The design of CLOB evolved from previous exchanges, Futures trading pits in Chicago reached a price discovery. 

For the purpose of price discovery, the match of buyers and seller needs to be atomic, otherwise one party cam pull out of the deal. 


\subsubsection*{Evolution of Exchange}
 

\subsubsection*{The HFT Problem}
\subsection*{Blockchain DEX Problems}
\subsubsection*{MEV and Public Blockchains}
\subsubsection*{Front-Running and Permissioned Blockchains}
\subsubsection*{Transaction Ordering}
\subsubsection*{Why does it not work?}
\subsection*{New Blockchain Abstraction}

\section{Fair Price Discovery} 

\subsection{Decentralized Limit Order Books}
We start with a closed network of exchange nodes, which each node has a matching-engine and maintains a CLOB, much like a centralized exchange. 

The distributed network of nodes come to consensus on the total ordering of transactions. This is done in two steps: 

\begin{enumerate}
    \item \emph{block-data} consensus is reached every N seconds on the full list of transactions from the mempool. 
    \item \emph{block-order} consensus is reached via auction on each block, where the highest bidder gets to reorder the transactions in the \emph{block-data} 
\end{enumerate}

From these two simple steps we have removed the advantage of HFT co-location, the need to front-run, and for \emph{Miner Extracted Value}

\paragraph*{HFT Co-Location} is no longer possible as there is no single location for the matching engine. Furthermore, the notion of \emph{being first} goes away, as all orders within the block are the same in regards to time. Also, since you can pay to \emph{be first} after the fact, there is no longer justification of cost of the HFT arms race. 

\paragraph*{Front-running} by \emph{rushing} has no advantage, as the processing order is not determined by which transaction was seen first on the network. 

\paragraph*{Miner Extracted Value} is no longer possible, as we come to consensus on the entire mempool, for each block. Recall, that \emph{MEV} is defined as a miner choosing, ordering or adding new transactions into a block in a way that generates economic value.  In our model, each block contains all the outstanding transactions, and the transactions in \emph{block-data} are in no particular order for processing, until after the \emph{block-order} auction. 

 
\newpage
\bibliography{sidepit}{}
\bibliographystyle{plain}
% \bibliographystyle{unsrt}

\end{document}